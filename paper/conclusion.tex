In this paper, we have addressed the issue of follow-up questions for inadequate bug reports. We have presented a novel approach of automatically selecting follow-up questions in such cases. To serve our purpose, we have constructed a dataset of 25K bug reports from 6452 unique repositories and generated 10 candidate follow-up questions for each each bug reports. We have used a neural networking approach to sort these questions and evaluated over model with four baselines. The results show that Bug-IQ model is encouraging to come up with a valid follow-up question. 

There are several avenues of future work. First, following Rao et al~\cite{rao2019answer}, we can automatically generate follow-up questions using sequence-to-sequence neutral network model~\cite{sutskever2014sequence, yin2015neural, serban2015building}. Second, Braslavski et al~\cite{10.1145/3020165.3022149}, we can work on generating frequently asked question patterns in bug reports. Third, we can develop a tool and integrate Bug-IQ model in real world platform like GitHub, BugZilla to help the users to point out what type of information may be missing from the original context.