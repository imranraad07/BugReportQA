This paper describes a technique for posing follow-up questions for incomplete bug reports that lack important information for triage, e.g., the bug's observable behavior. Our technique automatically selects follow-up questions from a corpus of such questions mined from the development histories of open source projects on GitHub. We select by using {\em tf*idf} to first identify a set of candidate follow-up questions whose original bug reports have high similarity to the deficient bug report of interest. Next, we use neural estimates of two metrics, compatibility and utility, to rank and select the optimal follow-up question to recommend. To evaluate our technique we curated a dataset of 25K bug reports from 6452 unique repositories and implemented four baselines. Our technique outperformed the baselines across the board, with a reasonable {\em Precision@1} score for our model of 0.49, i.e., nearly half of the top most recommended follow-up questions were considered valid. We also performed a survey of software developers which showed a follow-up question validity rate that aligned to the held-out dataset evaluation and also indicated that developers considered the selected follow-up questions as: useful, specific, and asking for new information not contained in the bug report.

There are several avenues of future work. First, following Rao et al.~\cite{rao2019answer}, we can attempt to automatically generate follow-up questions using sequence-to-sequence neutral network models~\cite{sutskever2014sequence, yin2015neural, serban2015building}. Second, following Braslavski et al.~\cite{10.1145/3020165.3022149}, we can work on generating frequently asked question patterns in bug reports. Third, we can develop a tool and integrate the \evpi model with real world platforms like GitHub or JIRA in order to assist developers in the field and gather more developer feedback. Lastly, another line of future work can be on broadening the evaluation, since it is a vital challenge to determine the most relevant follow-up question for different software development contexts and requirements.
