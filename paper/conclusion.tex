In this paper, we have addressed the issue of follow-up questions for inadequate bug reports. We have presented a novel approach of automatically selecting follow-up questions in such cases. To serve our purpose, we have constructed a dataset of 25K bug reports from 6452 unique repositories and generated 10 candidate follow-up questions for each bug reports. We have used a neural networking approach to sort these questions and evaluated over model with four baselines. The \textbf{P@1} score for our model is \textbf{0.49}, compared to random baseline's \textbf{0.319}. The human survey is aligned to the model's result. We can conclude that Bug-IQ model is encouraging to come up with a valid follow-up question.

There are several avenues of future work. First, following Rao et al~\cite{rao2019answer}, we can automatically generate follow-up questions using sequence-to-sequence neutral network model~\cite{sutskever2014sequence, yin2015neural, serban2015building}. Second, following Braslavski et al~\cite{10.1145/3020165.3022149}, we can work on generating frequently asked question patterns in bug reports. Third, we can develop a tool and integrate Bug-IQ model in real world platform like GitHub, BugZilla to help the users to point out what type of information may be missing from the original context. Lastly, another line of future work can be on evaluation. As it's possible to have multiple valid follow-up questions, a vital challenge will be determining the most relevant follow-up question based on different contexts and requirements.