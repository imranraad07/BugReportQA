\section{Evaluation}

We implemented a prototype of the system with the aim of evaluating its efficacy along
a few different dimensions. First, we use metrics and a held-out data set to evaluate the quality
of the recommendation, i.e., how well the system recommends valid follow-up questions for incomplete bug reports.
Second, we use a survey of software developers to evaluate the follow-up questions on their usefulness, novelty and specificity.


\subsection{Quality of Follow-Up Question Ranking}

One way of evaluating the ranking system, based on a held-out dataset (of bug reports and candidate follow-up questions), is
by using the posed questions as the ground truth. However, this simple setup has a serious deficiency in that the posed question
may not always be the most optimal among the set of candidate follow-up questions. More importantly, several of
the remaining candidate questions may be valid and (more) relevant to the bug report and therefore should
not be considered as negatively labeled instances for evaluation. Therefore, in order to provide an evaluation
set that identifies all of the valid questions in the candidate set, we perform manual annotation that clearly
identifies all of the valid follow-up questions for a specific bug report.

\subsubsection{Annotation}
We annotated 400 randomly chosen bug reports that were held out from our original corpus of 25K. The annotation
was performed by two of the authors following an agreed-upon predefined procedure. For each bug report, each annotator 1)
read the bug report carefully, spending a few minutes to understand its context, e.g., by looking at the purpose of the overall GitHub
project and the types of technologies it relies on; 2) marked all of the follow-up questions for the candidate set of 10
that were valid. Both of the annotators processed the same set of 400 bug reports, marking an average of 3.45/10 of the follow up questions as valid with an inter-annotator agreement (Cohen's kappa) of 0.60.
We use the set of follow-up questions that both annotators agreed were valid, i.e., the intersection between their annotation.

\subsubsection{Baselines}
The baselines we identified are meant to convey both straightforward approaches to ranking (e.g., directly using the Lucene output) and
ablation, i.e., using one part of our ranking function but not the other (e.g., ranking based only on the question utility, $U(q_{i})$).
We did not find appropriate prior research work to compare against, since the research direction is novel and models from other domains
with a similar purpose are too different in form. Below is an enumerated list of all of the ranking baselines we used.
\begin{itemize}
\item {\em Random} -- A random permutation of the candidate follow-up question list. We present metrics averaged over 10 runs.
\item {\em Lucene} -- Lucene uses the vector space model (i.e., tf*idf) to rank follow-up questions based on the similarity between the bug reports. This baseline just transfers Lucene's ranking, which we use to generate our candidate set of 10 follow-up questions, as the system's output.
\item {\em Utility only} -- $U(q_{i})$ -- The utility function, described in detail in Section~\ref{sec:ranking}, computes the average amount of OB,EB or S2R found in the answers to the specific follow-up question.
\item {\em Compatibility only} -- $P(q_{i}+a_{i}|br)$ -- The compatibility function computes the probability a bug report can be combined with a specific follow-up question and answer pair. The implementation uses a deep NN architecture to compute this quantity.
\end{itemize}

\subsubsection{Metrics}
We use a two popular information retrieval evaluation metrics: Mean Reciprocal Rank (MRR) and Precision@n (P@n).

The goal of MRR is to evaluate how effective is our technique, or a baseline, in locating the first valid follow-up question, as, presumably, this is a proxy for the ease with which an end-user would locate a follow-up question in the ranking. It is
computed as: $$MRR = \frac{1}{|B|} \sum_{i=1}^{|B|} \frac{1}{rank_{i}}$$ ,where $B$ is the set of bug reports in the test set and $rank_{i}$ is the ranked position of the first valid follow-up question for the $i^{th}$ bug report.

The goal of Precision@n is to measure the number of valid results when considering the top $n$ positions in the ranking. Unlike MRR, it consider all, not only the topmost ranked, results. It is computed as: $$P@n = \frac{1}{|B|} \sum_{i=1}^{|B|} \frac{|v|}{n}$$ ,where, as before, $B$ is the set of bug reports in the test set and $v$ is the set of valid follow-up questions ranked in the top $n$ positions. We use values of 1, 3 and 5 for $n$.


\begin{table}[t]
\centering
\caption{Evaluation results contrasting our system (\evpi) relative to several baselines.}
\begin{tabular}{p{3cm}cccc}
\hline
%                          & \multicolumn{4}{c}{$V_{1} \cap V_{2}$} \\\hline
                          & {\bf MRR}  & {\bf P@1}  & {\bf P@3}  & {\bf P@5}  \\\hline
{\em Random}              & 0.52 & 0.32 & 0.33 & 0.34 \\
{\em Lucene}              & 0.53 & 0.35 & 0.32 & 0.32 \\
{\em Utility only}        & 0.65 & 0.47 & 0.44 & 0.41 \\
{\em Compatibility only}  & 0.61 & 0.43 & 0.38 & 0.38 \\
{\em \evpi}                & 0.67 & 0.49 & 0.49 & 0.45 \\ \hline
\end{tabular}
\label{tab:results}
\end{table}


\subsubsection{Results} We summarize the results of our technique (\evpi) versus the identified
baselines in Table~\ref{tab:results}. Our results indicate that \evpi outperforms all of the baselines,
with the ablation-type baselines performing better than the simple baselines. The Lucene ranking
does surprisingly poor, basically in line with the Random baseline. The Utility only baseline is
the ones that comes closest to the performance of the full system. Perhaps the most intuitive result
is $P@1$, where \evpi scores 0.49, indicating that just about half of all of the top recommended follow-up
questions by our system were valid.


\subsection{Developer Survey}
While a recommended follow-up question may be valid, it may not possess other properties that would
encourage its use in a system that automatically poses follow-up questions for incomplete bug reports. For
instance, a follow-up question may be overly generic and lack detail or context specific to the bug report.
To investigate how our system performs across several such dimensions of interest we conducted a software developer survey.

%\subsubsection{Participants}
Through personal contacts, we e-mailed 8 software developers about our survey. None of the developers were
aware about the nature of our project outside of the content of our e-mail, which provided only
the most basic project context. After a positive response on our e-mail, we provided each developer with a
link to a Web form containing the survey. The participants were half from academia (graduate students at
institutions in the U.S. and Europe) and half professional developers from industry. All had programming
experience of 4 or more years with popular languages like Java or Python.

%Procedure
Each developer was assigned 12 instances of a bug reports and a follow-up question pair, where all of the follow-up questions
were the top-1 recommended by our \evpi system. 

%Measures

%Results


\subsection{Threats to Validity}
