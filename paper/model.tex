\section{System Description}

% describe overall system
As input, our system for retrieving follow-up questions requires: 1) an inadequate bug
report of interest; 2) a corpus of already posed follow-up questions extracted
from GitHub issues, including their corresponding bug reports and answers. As output, our system
produces a ranked list of follow-up questions appropriate to the inadequate bug report.
In this section, we describe our system, including how we create as a input
a large corpus of follow-up questions to recommend, how we select candidate follow-up questions from
this corpus for a specific incomplete bug report, and how we rank these questions in descending order of
their potential utility to the bug report.

\subsection{Selecting a Corpus of Bug Reports}


\begin{figure*}[ht]
\centering
\includegraphics[width=0.99\linewidth]{figures/pipeline.pdf}
\caption{Automatically curating a input corpus for our system follows a sequence of filtering steps, starting with filtering
GitHub repositories to selecting only the issues in these repositories that are bug reports with answered follow-up questions.}
\label{fig:sample_br}
\end{figure*}

% overview of the entire process
Our goal in curating a corpus of bug report-related follow-up questions and their answers
is to find a large, representative and high-quality corpus. Manually curated corpora are of
high quality but they are difficult to scale-up. Automatic curation can easily scale but it
can be affected by significant noise, leading to low data quality, unless care is taken
to filter and sample follow-up questions in a way that noise is mitigated. As corpus size
is an important factor in our system, we opt for an automated approach with numerous filters
to ensure the data is of highest possible quality. With the number of active repositories available
on GitHub providing a very large input domain, we can afford to err on the side of being overly
restrictive in our filtering.
To automatically curate our corpus, we: 1) select GitHub repositories that have high bug reporting activity,
as measured by the number of issues created by non-contributors over some fixed period of time; 2) select issues in those bug repositories that contain rapidly asked and succinct follow-up questions contained in GitHub issue comments; 3) locate
answers to the follow-up questions encoded as either comments or as edits to the
original bug report.

In more detail, we used the following sequence of steps to curate the corpus:
\begin{enumerate}
\item We scraped a set of public GitHub repositories with a high rate of
non-contributor created issues, where a non-contributor is a GitHub user that has never
committed any code in the specific repository. Repositories with these characteristics form
the target population of software projects that are more likely to use our technique. In order
to somewhat constrain  the amount of repositories, we focused on longer-running projects,
specifically, created between 2008-2014, and recently active with new issues created in the repository after Jan 1, 2019.
\item For each of these repositories, in their descending order of activity as defined in the previous step,
we selected issues in their GitHub issue tracker that are labeled as "bug", "crash", "fix", or unlabeled. Our goal for this step was
to avoid feature requests and focus on bug reports. We observed that issue labels were not used consistently enough
in projects, which is why we opted to include unlabeled issues.
\item We further selected only issues that contain follow-up questions in one of the issue comments.  We identified follow-up questions as GitHub comments containing only questions, identified by both starting with an interrogative word and ending with a question mark. In order to ensure we selected follow-up questions and not just
any questions, we constrained our selection based on time and comment sequence. That is, the comment containing the follow-up question must have been posted within 60 days of the issue creation date and must have occurred as the comment immediately following the post. We also ignored follow-up posts by the issue author by ensuring that the comment was authored by a different user from the author of the issue.
\item The set of issues and candidate follow-up questions from the previous step were further filtered to
retain issues and follow-up questions where an answer was provided. A key heuristic we used for recognizing an answer was that it was authored
by the original issue creator and occurred as the the next sequential comment
to the follow-up question. We also searched for answers that were encoded as edits to the original issue text by the author, which occurred after the follow-up question was posted and were limited to add at least 4 additional words to the original text in order to avoid minor spelling or grammar modifications.
\end{enumerate}

The highlights of the corpus curation process are also illustrated in Figure~\ref{fig:sample_br}. We stopped the collection process when we gathered a dataset of 25K GitHub issues, which we deemed to be sufficient for our purpose. Each data point in our dataset is a triple of
issue, follow-up question, and answer. Together, the 25K triples form the primary corpus we used to
retrieve and rank follow-up questions for a given inadequate bug report.


\subsection{Selecting Candidate Follow-Up Questions}

Selecting a set of most likely candidate follow-up questions for a specific incomplete bug report of interest from the corpus
of 25K triples (bug reports, follow-up questions and their answers) can be formulated as an information retrieval
problem. That is, as a query we can use the text of the incomplete bug report. We can represent the corpus
as an inverted index of the bug report text (e.g., using Lucene), and use tf*idf as the
ranking mechanism. In this way, we retrieve a set of 10 candidate follow-up questions for each incomplete bug report
of interest, where these 10 candidates have the most similar bug report text to the bug report of index.

More specifically, we create a Lucene index of the corpus of bug reports by following this set of steps:
\begin{itemize}
\item {\em Filtering.} Removing code or stack traces from bug reports in our index allows
for more accurate matching. GitHub issues use markdown so we remove all text surrounded
by triple-quotes as this is typically how source code and stack traces are encoded. We also
remove quoted text (i.e., lines that start with the greater than symbol) as these typically
refer to some external information, which, again, is often stack traces, code, or project documentation.
\item {\em Tokenizing.} We perform standard tokenization used in software engineering applications
of information retrieval~\cite{marcus2004information,shepherd2012sando}. We tokenize on white space, remove
punctuation (except for horizontal dashes)
and split on camel case and dashes, while also keeping the original unsplit tokens.
\item {\em Indexing.} We use Lucene's standard configuration to index the title and the body in separate fields,
in order to ensure better quality matches, especially when the bug report body has a lot of text. In order to provide
more context, we extract the tags that the bug report is labeled with (e.g., {\em fix-later, critical}) and the tags the
GitHub repository is tagged with (e.g., {\em java, linux, web-server}). The former provides specifics on the issue while the
latter usually denotes technologies or the project uses or broad categories it belongs to.
\end{itemize}

To create a query out of the incomplete bug report, we tokenize the its title and body
using the equivalent process to the one we performed on the bug reports in the corpus.
While obtaining a ranked list of follow-up questions from Lucene, we obtain and list of 10
triples and ignore the ranking. In the following section, we describe a customized ranking, which
takes more factors into account when choosing which follow-up question to pose to the inadequate
bug report.


\subsection{Ranking the Candidate Follow-Up Questions}

To rank the set of candidate follow-up questions for a specific bug report we reformulate
the notion of the Expected Value of Perfect Information (EVPI), initially proposed as a
neurally computed quantity by Rao et al., towards bug reports and their inadequacy. EVPI estimates
the value to the incomplete bug report of the answer to the follow-up question. We express
the EVPI of a specific follow-up question $q_{i}$ from our candidate set, given an inadequate
bug report $br$, as the product of the probability of obtaining a specific answer $a_{i}$ and
the utility that answer would provide to $br$, i.e., the utility of $br$ augmented by $a_{i}$.

\medskip
$EVPI(q_{i}|br) = P(a_{i}|q_{i},br) * U(br+a)$
\medskip

In other words, this formulation scales the utility of an answer to a follow-up question by the
likelihood we are to receive this answer, given the follow-up question and the bug report.
To compute the utility of the augmented bug report, $U(br+a)$, we resort to the pattern based
identification of the constituent pieces of a bug report (rxpected Behavior,...) proposed by Chaparro et al.
We implement 5 of the most common pattern from the pattern catalogue, which rely on key words
and part-of-speech tagging to ensure high precision.



%
% p(a_i|q_i,p_i) is estimated via a 5-layer feed-forward NN.
%
% U(p+a) = softmax(d_OB+d_EB+dS2R), where dOB = OBa_i - OBp, and similarly dEB and dS2R
%
% OB_x is the number of OB sentences in document x, as defined by Chaparro et al.
