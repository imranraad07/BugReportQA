\section{System Description}

% describe overall system
As input, our system for retrieving follow-up questions requires: 1) an (incomplete) bug
report of interest; 2) a corpus of already posed follow-up questions extracted
from GitHub issues; 3) answers to the each of the follow-up questions posed in (2). As output, our system
produces a ranked list of follow-up questions appropriate to the particular bug report.
In the remainder of this section, we describe how we create a large corpus of follow-up questions
to recommend, how we select questions from this corpus and how we rank these questions in order
of their potential utility to the bug report.

\subsection{Selecting a Corpus of Bug Reports}

% overview of the entire process
Our goal in curating a corpus of bug report-related follow-up questions and their answers
is to find a large, representative and high-quality corpus. Manually curated corpora are
high quality but they are difficult to scale-up. Automatic curation can be affected by
significant noise (e.g., overly specific or non-follow-up questions) unless care is taken
to filter and sample follow-up questions in a way that such noise is mitigated.

To automatically curate a corpus, we start by selecting GitHub repositories that have high bug reporting activity,
as measured by the number of issues created by non-contributors over some fixed period of time.
We select issues in those bug repositories that contain rapidly asked and succinct follow-up
questions, looking in the GitHub issue comments for these responses. Finally, we look for
a subsequent rapid answer to the follow-up question, encoded as either another comment or as an edit to the
original issue.

% list of more detailed steps to implement the plan
In more detail, we used the following set of steps and heuristics to curate the corpus:
\begin{enumerate}
\item We scraped a set of public GitHub repositories, sorted by the number of
non-contributor created issues, where a non-contributor is a GitHub user that has never
committed any code in the specific repository. In order to somewhat limit the amount of repositories,
we focused on longer-running projects, specifically, created between 2008-2014, and recently
active with new issues created in the repository after Jan 1, 2019.
\item For each repository in their sorted order from the previous step, we selected issues in the GitHub issue tracker
that contain follow-up questions as on of the issue comments.  Follow-up questions were identified as sentences starting with a question word and ending with a question mark. In order to ensure we selected follow-up questions and not just
any questions, we filtered based on time. That is, the comment containing the follow-up question must be posted
within 60 days of the issue creation date and must occur as one of the first 3 comments to the post. Just as an additional safety mechanism,
we ensured that the comment was authored by a different user from the author of the issue.
\item The set of issues and candidate follow-up questions from the previous step were further filtered to
ensure an answer was provided. A key characteristic of the answer that we searched for is that it was authored
by the original issue creator. We allowed for answers to be posted as comments, as long as they occurred as the next sequential comment
to the follow-up question. We also allowed for answers to be encoded as edits to the original issue text, which we limited to add at least 4 additional words to avoid minor spelling or grammar modifications.
\item {\bf TODO: how can we claim these were bugs and not feature requests?}
\end{enumerate}

Finally, we stopped the collection process when we gathered a dataset of N GitHub issues. These triples of
issue, follow-up question, and answer formed the primary corpus we used to retrieve a rank follow-up questions
for a given incomplete bug report.

\subsection{Selecting Candidate Follow-Up Questions}

Given the corpus of follow-up questions and an incomplete bug report, selecting the
most likely candidate follow-up questions can be formed as an information retrieval
problem. That is, given the incomplete bug report as a query and an inverted index
created from the corpus (using Lucene), using tf*idf as the ranking mechanism, we
retrieve a set of 10 candidate follow-up questions, including also their original issue
and answer.

More specifically, we create a Lucene index by following the following set of steps
on the corpus.
\begin{itemize}
\item
\end{itemize}

To create a query out of the incomplete bug report, we do the following:
\begin{itemize}
\item
\end{itemize}


\subsection{Ranking Follow-Up Questions}

% EVPI(q_i|p) = p(a_i|q_i,p_i) * U(p+a)
%
% p(a_i|q_i,p_i) is estimated via a 5-layer feed-forward NN.
%
% U(p+a) = softmax(d_OB+d_EB+dS2R), where dOB = OBa_i - OBp, and similarly dEB and dS2R
%
% OB_x is the number of OB sentences in document x, as defined by Chaparro et al.
