\section{System Description}

% describe overall system
As input, our system for retrieving follow-up questions requires: 1) an (incomplete) bug
report of interest; 2) a corpus of already posed follow-up questions extracted
from GitHub issues; 3) answers to the each of the follow-up questions posed in (2). As output, our system
produces a ranked list of follow-up questions appropriate to the particular bug report.
In the remainder of this section, we describe how we create a large corpus of follow-up questions
to recommend, how we select questions from this corpus and how we rank these questions in order
of their potential utility to the bug report.

\subsection{Selecting a Corpus of Bug Reports}

% overview of the entire process
Our goal in curating a corpus of bug report-related follow-up questions and their answers
is to find a large, representative, high-quality corpus. Manually curated corpora are high quality but they
are difficult to scale-up and to evolve over time. Automatic curation can be affected by
significant noise (e.g., overly specific or non-follow-up questions) unless care is taken
to filter and sample in a way that such noise is mitigated.

To ensure  GitHub repositories that have high bug reporting activity, as measured by the number of
issues created in the tracker over some fixed period of time by

% list of more detailed steps to implement the plan



\subsection{Selecting Candidate Follow-Up Questions}

% <Lucene>

\subsection{Ranking Candidate Follow-Up Questions}

% EVPI(q_i|p) = p(a_i|q_i,p_i) * U(p+a)
%
% p(a_i|q_i,p_i) is estimated via a 5-layer feed-forward NN.
%
% U(p+a) = softmax(d_OB+d_EB+dS2R), where dOB = OBa_i - OBp, and similarly dEB and dS2R
%
% OB_x is the number of OB sentences in document x, as defined by Chaparro et al.
