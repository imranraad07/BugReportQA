\section{Related Work}
To our knowledge, ours is the first effort towards leveraging follow-up questions for
improving bug report quality. The related research can be grouped into two categories, techniques
for determining and improving bug report quality, and uses of follow-up questions in other domains,
external to bug reports and software engineering.

%bug report quality (focus first on measuring, then on improving)
\noindent
{\em Analyzing the quality of bug reports.} The quality of user written bug reports,
in the era of open source development, is a topic several researchers have been interested
in. Bettenburg et al. proposed techniques for automatically identifying stack traces, code
snippets, and other structures in bug reports~\cite{bettenburg08extracting}. Linstead and Baldi
applied LDA to a large corpus of bug reports to determine semantic coherence of these
documents~\cite{linstead09mining}. Davies et al. manually analyzed a corpus of bug reports
from four popular open source projects observing
that observable behavior and expected behavior were the most consistently found parts
of a bug report~\cite{davies14whats}. Using surveys, researchers observed that developers
consider steps to reproduce, test cases and stack traces most helpful in bug reports, but
many of those parts were also the hardest for users to supply. Chaparro et al. developed a
technique to automatically extract observable behavior, expected behavior, and the steps to
reproduce from a bug report~\cite{chaparro17detecting}

\noindent
{\em Improving bug report quality.} Researchers have approached improving the quality of
bug reports along a few different dimensions. One line of work, with numerous techniques,
is to detect duplicate bug reports~\cite{chaparro19reformulating}. Another research avenue
is to classify bug reports into valid vs. invalid or easy vs. difficult~\cite{fan20chaff,hooimeijer07modeling}.
More recently, researchers have attempted to automatically improve parts of bug reports. Moran et al. provided
auto-completion for the steps to reproduce portion of bug reports by leveraging image processing
of screenshots taken from the application's UI~\cite{moran15autocompleting}.

\noindent
{\em Automatically posing follow-up questions.}
